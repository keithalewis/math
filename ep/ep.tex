\newcommand\RR{\mathbf{R}}
\newcommand\BB{\mathcal{B}}
\newcommand\LL{\mathcal{L}}
\newcommand\Var{\operatorname{Var}}
\newcommand\Cov{\operatorname{Cov}}

Given two random realized returns on an investment, which is to be
preferred? This is a fundamental problem in finance that has no
definitive answer except in the case one investment always returns more
than the other, in which case arbitrage exists. In 1952
Markowitz{[}@Mar1952{]} and Roy{[}@Roy1952{]} introduced a criterion for
risk vs.~return in portfolio selection: if two portfolios have the same
expected realized return then prefer the one with smaller variance. An
efficient portfolio has the least variance among all portfolios having
the same expected realized return. This was developed into the Capital
Asset Pricing Model by Treynor{[}@Tre1961{]}, Sharpe{[}@Sha1964{]},
Lintner{[}@Lin1965{]}, and many others.

The Capital Asset Pricing Model marked the transition from the due
diligence required for Graham-Todd security analysis to using the wisdom
of the markets to inform investing. The ``market portfolio'' was assumed
to be in an efficient ``equilibrium'' resulting from the cadre of
investment professionals performing ``market clearing'' trades. This
short note is agnostic to the quoted terms and proves a simple
mathematical result about efficient portfolios.

There are well-founded criticisms of the CAPM, but it has value as an
easily understood model. Portfolio managers use Sharpe ratios to tailor
returns for an investment strategy while accounting for risk. The CAPM
demonstrates a constraint on expected returns and covariance of
efficient portfolios. We show a much stronger constraint: efficient
portfolios satisfy an equality of realized returns as random variables.
This allows the value-at-risk, or any risk measure, of efficient
portfolios to be calculated, something not possible using the classical
result that only holds for expected values.

This result follows directly from writing down a mathematical model for
one period investments. The only thing remarkable is that this has not
heretofore been noted in the literature. Prior work fails to explicitly
specify a sample space and probability measure, the first step in any
model involving probability since Kolomogorov legitimized probability as
a branch of measure theory {[}@Kol1956{]}.

\subsection{CAPM}\label{capm}

The CAPM places a constraint on the excess expected realized return of
efficient portfolios. \[
\tag{1} E[R] - R_0 = \beta(E[R_1] - R_0)
\] where \(R\) is the realized return of an efficient portfolio, \(R_0\)
is the realized return of a risk-less portfolio, \(R_1\) is the realized
return of the ``market portfolio'', and
\(\beta = \operatorname{Cov}(R, R_1)/\operatorname{Var}(R_1)\).

This short note shows the random realized return \(R\) of any efficient
portfolio satisfies \[
\tag{2} R - R_0 = \beta(R_1 - R_0)
\] where \(R_0\) and \(R_1\) are the random realized returns of any two
independent efficient portfolios. This implies
\(\beta = \operatorname{Cov}(R - R_0, R_1 - R_0)/\operatorname{Var}(R_1 - R_0)\).
Taking expected values of both sides when \(R_0\) has zero variance and
\(R_1\) is the ``market portfolio'' yields the classical CAPM formula

\subsection{One-Period Model}\label{one-period-model}

Let \(I\) be the set of \emph{market instruments} and \(\Omega\) be the
set of possible \emph{market outcomes} over the period. The
\emph{one-period model} specifies the initial instrument prices
\(x\in\mathbf{R}^I\) and the final instrument prices
\(X\colon\Omega\to\mathbf{R}^I\) depending on the outcome
\(\omega\in\Omega\) that occurs. The one period model also specifies a
probability measure on the space of outcomes. It is common to assume
\(\Omega\) is \(\mathbf{R}^I\), \(X\) is the identity function, and
\(P\) is multivariate normal. We allow arbitrary distributions to be
specified for final prices.

A \emph{portfolio} \(\xi\in\mathbf{R}^I\) is the number of shares
initially purchased in each instrument. It costs
\({\xi^* x = \sum_{i\in I} \xi_i x_i}\) to acquire the portfolio at the
beginning of the period and returns
\({\xi^* X(\omega) = \sum_{i\in I} \xi_i X_i(\omega)}\) when liquidated
at the end of the period if \(\omega\in\Omega\) occurs. The
\emph{realized return} of \(\xi\) is \({R_\xi = \xi^* X/\xi^* x}\) when
\(\xi^* x \not= 0\).

\subsection{Efficient Portfolio}\label{efficient-portfolio}

A portfolio is \emph{efficient} if its variance is less than or equal to
the variance of any portfolio having the same expected realized return.
Note \(R_\xi = R_{t\xi}\) for any non-zero \(t\in\mathbf{R}\) so there
is no loss in assuming \(\xi^* x = 1\). In this case \(R_\xi = \xi^* X\)
is the realized return of the portfolio. If \(\xi^* x = 1\) then the
variance of the realized return is
\(\operatorname{Var}(R_\xi) = \xi^*V\xi\) where \({V = E[X X^*] -
E[X] E[X^*]}\).

For a given expected realized return \(r\in\mathbf{R}\) we can use
Lagrange multipliers to minimize
\({\frac{1}{2}\xi^* V\xi - \lambda(\xi^* x - 1) - \mu(\xi^* E[X] - r)}\)
over \(\xi\in\mathbf{R}^I\), \({\lambda, \mu\in\mathbf{R}}\). As is
well-known, the first order condition on \(\xi\) is
\({0 = V\xi - \lambda x - \mu E[X]}\). See the
\hyperref[appendix]{Appendix} for a proof.

If \(V\) is invertable then \[
    \xi =  \lambda V^{-1}x + \mu V^{-1} E[X].
\] This shows every efficient portfolio is in the span of \(V^{-1}x\)
and \(V^{-1} E[X]\).

The only novel result in this paper is the observation that if \(\xi_0\)
and \(\xi_1\) are any two independent efficient portfolios then
\({\xi = \beta_0\xi_0 + \beta_1\xi_1}\) for some scalars \(\beta_0\) and
\(\beta_1\) so
\({R_\xi = (\beta_0 R_{\xi_0} + \beta_1 R_{\xi_1})/(\beta_0 + \beta_1)}\).
This shows \[
    R_\xi - R_{\xi_0} = \beta(R_{\xi_1} - R_{\xi_0})
\] as random variables where \(\beta = \beta_1/(\beta_0 + \beta_1)\).
Taking the covariance with \({R_{\xi_1} - R_{\xi_0}}\) on both sides
gives \[
    \beta = \operatorname{Cov}(R_\xi - R_{\xi_0}, R_{\xi_1} - R_{\xi_0})/\operatorname{Var}(R_{\xi_1} - R_{\xi_0}).
\]

If \(V\) is not invertable then there exists \(\zeta\in\mathbf{R}^I\)
with \({V\zeta = 0}\). The first order condition
\({0 = -\lambda x - \mu E[X]}\) gives \(x = (-\mu/\lambda)E[X]\). The
first order conditions \({0 = \zeta^*x - 1}\), and
\({0 = \zeta^*E[X] - r}\) show \({1 = (-\mu/\lambda)r}\) so
\({x = (1/r)E[X]}\). This is a special case of the condition for a
one-period model to be \hyperref[ftap]{arbitrage-free}.

There may be two independent portfolios having variance zero. If they
have different returns then arbitrage exists. If they have the same
return then the model has redundant assets.

\subsection{Appendix}\label{appendix}

We use the notation \(\xi^*\) for what is usually denoted by the
transpose \(\xi^T\) or \(\xi'\). It is simpler and more illuminating to
work with abstract vector spaces and linear operators between them than
with \(\mathbf{R}^n\) and matrices. Matrix multiplication is just
composition of linear operators.

Recall \(\mathbf{R}^I = \{x\colon I\to\mathbf{R}\}\) is the vector space
of all functions from the set \(I\) to \(\mathbf{R}\) with scalar
multiplication and vector addition defined point-wise:
\({(ax)(i) = ax(i)}\) and \({(x + y)(i) = x(i) y(i)}\) for
\(a\in\mathbf{R}\), \({x,y\in\mathbf{R}^I}\), and \(i\in I\).

For \(\xi\in\mathbf{R}^I\) define
\(\xi^*\colon\mathbf{R}^I\to\mathbf{R}\) by
\(\xi(x) = \xi^*x = \sum_{i\in I} \xi_i x_i\) if \(I\) is finite. Note
\(\xi^*\) is linear.

Let \(\mathcal{L}(V,W)\) be the set of all linear operators from the
vector space \(V\) to \(W\). Note \(\mathcal{L}(V,W)\) is also a vector
space with scalar multiplication and addition defined point-wise. The
dual of a vector space \(V\) is \(V^*=\mathcal{L}(V,\mathbf{R})\). For
\(\xi\in\mathbf{R}^I\) we have \(\xi^*\in (\mathbf{R}^I)^*\) and
\(\xi^*x = x\xi^*\) allows us to identify \((\mathbf{R}^I)^*\) with
\(\mathbf{R}^I\). If \(T\in\mathcal{L}(V,W)\) its adjoint is
\(T^*\in\mathcal{L}(W^*,V^*)\) defined by \(T^*w^*\in V^*\) where
\(T^*w^*(v) = w^*(Tv)\), \(w^*\in W^*\), \(v\in V\). If
\(S\in\mathcal{L}(W,U)\) then \(ST\in\mathcal{L}(V,U)\) and
\((ST)^* = T^*S^*\).

Let \(\mathcal{B}(V,W)\) be the set of bounded linear operators from the
normed linear spaces \(V\) to \(W\). A linear operator
\(T\in\mathcal{L}(V,W)\) is bounded if there exists \(C\in\mathbf{R}\)
with \(\|Tv\| \le C\|v\|\) for all \(v\in V\). The least upper bound of
such constants is the norm of \(T\). This makes \(\mathcal{B}(V,W)\) a
normed vector space.

\subsubsection{Fréchet derivative}\label{fruxe9chet-derivative}

If \(F\colon V\to W\) is a function between normed vector spaces its
Fréchet derivative \({DF\colon V\to \mathcal{B}(V,W)}\) is defined by \[
    F(\xi + h) = F(\xi) + DF(\xi)h + o(\|h\|)
\] where \(f(h) = g(h) + o(\|h\|)\) means
\((\|f(h) - \|g(h)\|)/\|h\|\to 0\) as \(\|h\|\to 0\). If the Fréchet
derivative exists at \(\xi\) then \(F\) can be approximated by a linear
operator near \(\xi\).

Given \(c\in\mathbf{R}^I\) define \(F\colon\mathbf{R}^I\to\mathbf{R}\)
by \(F(\xi) = \xi^*c\). We have \({F(\xi + h) = \xi^*c + h^*c}\) so
\(DF(\xi) = c^*\) since \(h^*c = c^*h\).

Given \(T\colon\mathbf{R}^I\to\mathbf{R}^I\) define
\(F\colon\mathbf{R}^I\to\mathbf{R}\) by \(F(\xi) = \xi^*T\xi\). We have
\[
\begin{aligned}
    F(\xi + h) &= (\xi + h)^*T(\xi + h) \\
        &= \xi^*T\xi + \xi^*Th + h^*T\xi + h^*h \\
        &= \xi^*T\xi + \xi^*Th + \xi^*T^*h + o(\|h\|). \\
\end{aligned}
\] This shows \(DF(\xi) = \xi^*(T + T^*)\) so \(DF(\xi) = \xi^* 2T\) if
\(T* = T\).

\subsubsection{Lagrange Multiplier}\label{lagrange-multiplier}

To find the minimum value of \(\operatorname{Var}(R_\xi)\) given
\(E[R_\xi] = r\) we use Lagrange multipliers and solve \[
        \min \frac{1}{2}\xi^* V\xi - \lambda(\xi^* x - 1) - \mu(\xi^* E[X] - r)
\] for \(\xi\in\mathbf{R}^I\), \(\lambda, \mu\in\mathbf{R}\). If
\({\xi^* \xi = 1}\) then \({R_\xi = \xi^* E[X]}\) and
\({\operatorname{Var}(R_\xi) = \xi^* V\xi}\) where
\({V = E[XX^*] - E[X]E[X^*]}\).

Since \(V^* = V\), the first order conditions for an extremum are \[
\begin{aligned}
        0 &= \xi^*V - \lambda x^* - \mu E[X^*] \\
        0 &= \xi^* x - 1 \\
        0 &= \xi^* E[X] - r \\
\end{aligned}
\] Assuming \(V\) is left invertable
\(\xi = V^{-1}(\lambda x + \mu E[X])\). Note every extremum lies in the
(at most) two dimensional subspace spanned by \(V^{-1}x\) and
\(V^{-1}E[X]\).

The constraints \(1 = x^*\xi\) and \(r = E[X^*]\xi\) can be written \[
\begin{bmatrix}
1 \\
r \\
\end{bmatrix}
=
\begin{bmatrix}
\lambda x^*V^{-1}x + \mu x^*V^{-1}E[X] \\
\lambda E[X^*]V^{-1}x + \mu E[X^*]V^{-1}E[X] \\
\end{bmatrix}
= \begin{bmatrix}
A & B \\
B & C\\
\end{bmatrix}
\begin{bmatrix}
\lambda \\
\mu
\end{bmatrix}
\] with \(A = x^* V^{-1}x\), \(B = x^* V^{-1}E[X] = E[X^*]V^{-1}x\), and
\(C = E[X^*] V^{-1}E[X]\). Inverting gives \[
\begin{bmatrix} \lambda \\ \mu \end{bmatrix}
= \frac{1}{D}
\begin{bmatrix}
C & -B \\
-B & A\\
\end{bmatrix}
\begin{bmatrix}
1 \\
r
\end{bmatrix}
=
\begin{bmatrix}
(C - r B)/D \\
(-B + r A)/D\\
\end{bmatrix}
\] where \(D = AC - B^2\). The solution is \(\lambda = (C - r B)/D\),
\(\mu = (-B + r A)/D\), and \[
    \xi = \frac{C - r B}{D} V^{-1}x + \frac{-B + r A}{D} V^{-1}E[X].
\]

A straightforward calculation shows the variance is \[
\operatorname{Var}(R_\xi) = \xi^* V\xi = (C - 2Br + Ar^2)/D.
\]

\subsubsection{FTAP}\label{ftap}

Arbitrage exists in the one-period model if there is a
\(\xi\in\mathbf{R}^I\) with \(\xi^* x < 0\) and \(\xi^* X(\omega)\ge0\)
for \(\omega\in\Omega\). The cost of putting on a position
\(\xi\in\mathbf{R}^I\) is \(\xi^*x\) so you make money entering the
position and never lose money unwinding it.

Note if \(x = \sum_j X(\omega_j) D_j\) where \(\omega_j\in\Omega\) and
\(D_j\) are non-negative scalars then
\({\xi^*x = \sum_j \xi^*X(\omega_j) D_j \ge 0}\) if \(\xi^*X\ge0\). In
this case there is no arbitrage.

The one-period Fundamental Theorem of Asset Pricing states there is no
model arbitrage if and only if \(x\) belongs to the smallest closed cone
containing the range of \(X\). Note this statement does not involve any
measures. The FTAP is a geometric result, not a probabilistic result.

Recall that a \emph{cone} is a subset of a vector space closed under
addition and multiplication by a positive scalar, that is,
\(C + C\subseteq C\) and \(tC\subseteq C\) for \(t > 0\). For example,
the set of arbitrage portfolios is a cone.

The above proves the ``easy'' direction. The contra-positive follows
from the

\textbf{Lemma.} \emph{If \(x\in\mathbf{R}^n\) and \(C\) is a closed cone
in \(\mathbf{R}^n\) with \(x\not\in C\) then there exists
\(\xi\in\mathbf{R}^n\) with \(\xi^* x < 0\) and \(\xi^* y \ge0\) for
\(y\in C\).}

\emph{Proof.} Since \(C\) is closed and convex there exists nearest
\(\hat{x}\in C\) with \(0 < \|\hat{x} - x\| \le \|y - x\|\) for all
\(y\in C\). Let \(\xi = \hat{x} - x\). For any \(y\in C\) and \(t > 0\)
we have \(ty + \hat{x}\in C\) so \(\|\xi\| \le \|ty + \xi\|\).
Simplifying gives \(t^2\|y\|^2 + 2t\xi^* y\ge 0\). Dividing by \(t > 0\)
and letting \(t\) decrease to 0 shows \(\xi^* y\ge 0\). Take
\(y = \hat{x}\) then \(t\hat{x} + \hat{x}\in C\) for \(t \ge -1\). By
similar reasoning, letting \(t\) increase to 0 shows
\(\xi^* \hat{x}\le 0\) so \(\xi^* \hat{x} = 0\). Because
\(0 < \|\xi\|^2 = \xi^* (\hat{x} - x) = -\xi^* x\) we have
\(\xi^* x < 0\).

The proof also shows \(\xi\) is an arbitrage when one exists.

If \(X\) is bounded, as it is in the real world, then there exists a
positive finitely-additive measure {[}@DunSch1958{]} with
\(x = \int_\Omega X\,dD\). Since \(D/D(\Omega)\) is a positive measure
with mass 1 we have \(x = E[X]D(\Omega)\) under this ``probability''
measure.

We say \(\zeta\in\mathbf{R}^I\) is a zero coupon bond if
\(\zeta^* X = 1\). Since \(\zeta^*x = \int_\Omega dD\) the realized
return on \(\zeta\) is is the constant \(R_\zeta = 1/D(\Omega)\). The
\emph{discount} of the zero coupon bond is \(D(\Omega) = 1/R_\zeta\). In
this case \(x\) is the discounted ``expected value'' of \(X\).

\subsection{References}\label{references}
